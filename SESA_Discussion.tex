\section{Discussion}

Our results show that the tendency of southbound semipalmated sandpipers to aggregate at the safest stopover locations has steadily increased since 1974. Sensitivity analyses establish that the shift appears to have been made specifically toward sites of higher safety, rather than to larger sites. The shift cannot be accounted for by inclusion (or exclusion) of the two sites exerting most leverage (Mary's Point and Johnson's Mills; see \autoref{tab:leverage}), by possible confounds arising from habitat reduction, by the selection of survey dates included in the analysis, or by altering our assumptions regarding the definitions of site size and danger (Figure S2).

We predicted this shift based on the well-established increase in continental falcon populations since the early 1970s. A similar redistribution was previously documented for wintering dunlins on North America’s Pacific coast \citep{Ydenberg2017} and was also attributed to the large increase in falcon presence. Of particular note is the introduction of captive-reared peregrines in the 1970s and ‘80s to major stopover areas such as the Bay of Fundy \citep{Dekker2011} and Delaware and Chesapeake Bays and their environs \citep{Watts2015}. With home ranges of 123 -- 1175 km$^2$ and a daily range of 23 km$^2$ around breeding locations \citep{Enderson1997a,Jenkins1998,Ganusevich2004}, the impact of breeding peregrines musty be widespread throughout both regions.

For most of the twentieth century, these regions were essentially predator-free during sandpiper passage, so stopover site choices and behaviour by migrant sandpipers could have been based primarily on food availability, with the danger posed by falcons ignored. \citet{Lank1983} observed individual semipalmated sandpipers at Kent Island in the Bay of Fundy during the late 1970s so encumbered by fat that they were captured by gulls. Paralleling observations made on western sandpipers in the Strait of Georgia \citep{ydenberg_western_2004}, semipalmated sandpiper fuel loads in the Bay of Fundy have decreased at small, dangerous locations such as Kent Island, but not at large, safe locations such as Johnson's Mills \citep{hope2010influence}. The mass decline is attributed to the reduced predator escape performance induced by large fuel loads \citep{burns_effects_2002}, and is consistent with the hypothesis that stopover site choice and behaviour is strongly influenced by the trade off between fuel loading and predation danger \citep{Pomeroy2008a,Taylor2007}.  

Migrant sandpipers have previously been shown to be sensitive to predation danger on migration. The migratory behaviours responding to danger include flock size, vigilance, over-ocean flocking during high tides, length of stay at dangerous locations, location selection, habitat selection within a location, and fuel load \citep{Dekker1998,ydenberg_western_2004,pomeroy_tradeoffs_2006,Pomeroy2008a,Sprague2008a}. Migrant sandpipers also change their behaviour seasonally in relation to their temporal proximity to the arrival of migrant peregrines \citep{Hope2014,Hope2011}. In a previous paper \citep{Lank2017} we attributed the shortening wing length measured 1980 - 2015 in semipalmated sandpipers and other calidridines \citep{wingcalidris2019} to selection for better predator escape performance \citep[see also ][]{ydenberg_hope2019}.

The PMD decline (\autoref{fig:temporal-plot}) has progressed steadily since 1974. The decline in PMD arose as the usage shown in \autoref{fig:example-plot}A shifted rightward, reducing the index value by 0.4\% per year, for a total decline of 18\% since 1974. It might be expected that the higher rate of increase in the number of breeding falcons in more recent years (\autoref{fig:fig-1}) should have accelerated the PMD decline. But the large majority of intertidal feeding area is on safer sites (see \autoref{fig:example-plot}B), with dangerous sites contributing disproportionately little to the total. We hypothesize that the initial small number of peregrines had a very large effect at small, dangerous locations \citep[e.g. ][]{page1975raptor}, where usage presumably began to drop in earlier years. The impact of additional peregrines was reduced as usage shifted to larger and safer sites. 

Larger groups also have benefits in reducing the likelihood of being selected by a predator (dilution), and increased detection of predator attacks \citep[many eyes][]{Roberts1996,Bednekoff1998,Fernandez-Juricic2007,Pays2013}. With predation dilution can also come increased competition during foraging \citep{Stillman1997,Vahl2005,Minderman2006c}. While the Bay of Fundy provides rich and widespread food for refuelling sandpipers, competitive interactions that reduce foraging efficiency likely occur at small scales \citep{Vahl2005a,Beauchamp2009a,Beauchamp2014}. For most semipalmated sandpipers, the benefits of large aggregations appear to outweigh the costs to foraging efficiency. 

Shorebird population census work indicates that many species, among them semipalmated sandpipers, have declined steadily since the 1970s \citep{bart_survey_2007,Andres2012b,Gratto-Trevor2012,Smith2012a,morrison_dramatic_2012}. Could the shift to safer stopover sites observed here be driven by this population decline? The PMD index is calculated based on proportions, so a reduction in sandpiper numbers would not affect the PMD value unless accompanied by a distributional change. Any distribution that includes safety considerations \citep{grand1999predation,Moody1996} would be expected to shift the towards safer sites as numbers decline even without an increase in predator numbers, due to the heightened danger of smaller numbers. The shift should progress until the fitness costs (reduced feeding rate) of the safer sites is compensated by the benefit (increased safety), which in turn depends on the marginal rates of change in food and safety with sandpiper density at each location \citep{Ydenberg2017}. Further evaluation is required.

Our analysis confirms the measured shift in the PMD index is better explained by a shift in distributions specifically toward safer and not just toward larger sites. The mean census numbers in the dataset used here show no temporal trend at all (see Results), but there has been a well-documented establishment of a large locally breeding population of peregrines. We are unable to exclude a possible contribution from population decline to the PMD change measured here, but all the evidence available is consistent with a strong non-lethal influence of predation danger.

An increase in food availability, or a reduction in energy demand would also allow greater aggregation on larger and safer sites, and thus a shift away from smaller and more dangerous locations. There is so far as we are aware no evidence for any trend in food abundance in the Bay of Fundy, nor is there any change proposed by the literature. The copepod \textit{Corophium volutator}, is a major prey item for semipalmated sandpipers in the Bay of Fundy, and appears to vary in abundance between locations and within each year, but variation between years does not appear to be substantial \citep{Barbeau2009a}. Other studies have shown variation between years when looking at a wider array of potential food sources \citep{Quinn2012a}, but it appears that semipalmated sandpipers have flexibility in their food sources \citep{Quinn2017}, so that a decline at one location could be compensated by  increases at others.

Other mechanisms could affect sandpiper energy requirements and thus affect distributions by reducing the need for food. A temperature increase due to climate change could reduce existence energy, though we note that the great majority of the intake of semipalmated sandpipers is used as fuel for the long trans-Atlantic flight to South America --- which is not temperature dependent. Another possible climate change effect could operate by ecological mismatch \citep{Jones2010}. Southward migration timing is widely believed to match food at stopovers, but if the timing of the food peak has shifted due to climate change, the availability of food at stopovers could be affected. The result would be lower rather than higher food availability, which would, according to our current understanding, shift sandpipers to higher food (more dangerous, smaller) sites, opposite to the trend documented here.


%In a mortality-minimizing species, the observed shift to safer sites could be driven by a population decline alone. Recent analyses of shorebird population censuses including breeding, migration as well as non-breeding surveys all point to decreases since the 1970s in many species, including semipalmated sandpipers \citep{bart_survey_2007,Andres2012b,Gratto-Trevor2012,Smith2012a,morrison_dramatic_2012}. An ideal free distribution that includes prioritizing safety \citep{grand1999predation,Moody1996} would be expected to shift towards safer sites as sandpipers numbers declined even without an increase in danger. Our analysis confirms the measured shift in the PMD is due to a shift in distributions toward safer and not just larger sites. The decline in sandpiper censuses is not reflected in the census numbers in the dataset used here (see Results). However, as we have strong evidence for the recovery of peregrine falcons and evidence for a decline in the overall population of sandpipers, we conclude either factors could in theory drive the shift, but more likely both factors together would  increase the shift towards safer sites. We expect that the increase in predator abundance should be the stronger of the two, but disentangling the two effects is not possible in this analysis.

%The decline in sandpiper censuses is not reflected in the census numbers in the dataset used here (see Results), but of course may have occurred at locations not censused in any year, and so have affected the overall level of competition. %A population decline (or increase) would not change the value of the PMD index unless accompanied by a change in the distribution of usage across sites, because the PMD index depends only on the distribution, and is independent of the numbers of birds.  Hence a population decline could not on its own explain the PMD trend reported here.  Food availability would also increase if the number of sandpipers declined so that there were fewer competitors.

%Simulation modelling of other migratory systems has suggested that a population decline does not necessarily lead to redistribution \citep{Taylor2007}, if  stopover densities of birds are below the sites’ carrying capacities. A population decline could alter the food/safety trade-off at different sites by affecting dilution benefits or the intensity of food competition. But depending on the distribution of the quality (size, danger and food density) of stopover sites, this could drive the PMD index either up or down \citep[see discussion on redistribution in ][]{Ydenberg2017}. Further evaluation is required.


In conclusion, semipalmated sandpipers aggregate in large numbers at a few large and safe sites and have steadily shifted towards safer sites between 1974 and 2017. The results appear robust to various biases possibly inherent in the dataset, and we suggest that the observed trend as a response to increased predator populations, especially the (re)introduction of predators at major stopover areas along the southbound migratory route. This result matches the previously reported shift in the non-breeding distribution of Pacific dunlins over the same period \citep{Ydenberg2017}, as well as the reduced stopover duration of southbound western sandpipers at dangerous stopover sites \citep{ydenberg_western_2004}. These adjustments likely have consequences for the schedule and
routing of migration that we suspect may in turn contribute to substantial non-lethal effects on their populations.  