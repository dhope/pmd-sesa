\section{Introduction}

Predators affect prey populations not only by `direct killing' \citep[also termed `lethal',  `consumptive', `density-mediated' or `mortality' effects; ][]{christianson2014ecosystem}, but by inducing prey to adjust behaviour, physiology, morphology and life history to mitigate the danger \citep{moll2017many}. The adjustments are made behaviourally or by adaptive plasticity \citep[e.g.][]{domenici2007predator}, and through natural selection \citep[e.g.][]{reznick1990experimentally}. These `non-lethal' (also termed `non-consumptive', `trait-mediated', `intimidation' or `fear') effects on prey populations can be as strong or even stronger than the effects of direct killing, and together they readily propagate to have further effects on other trophic levels \citep{Terborgh2010,ohgushi2012trait}.


The effects of predators on prey populations are beginning to be examined at large scales in nature. \citet{madin2016human} and \citet{myers2003rapid} consider the effects on marine fish populations stemming from the great reductions in the abundance of top predators, while \citet{heithaus2008predicting} document many novel repercussions of changing marine predator abundances \citep[see also ][]{estes2011trophic,babcock2010decadal}. Still, much of what is known about these relationships between predators and prey relates to lethal effects (i.e. mortality inflicted directly by predators). In contrast, the non-lethal effects \citep[e.g. influences of predator presence on prey behaviour and morphology; see ][Box 1]{madin2016human} are not as well studied, despite many advances in the past two decades from the experimental literature \citep[reviewed by ][]{long2012impact}.

In contrast to the ongoing reductions in marine systems, top predators are currently increasing in abundance in some terrestrial systems. In the case of the peregrine falcon (\textit{Falco peregrinus}), migrating, wintering, and breeding numbers all show substantial, ongoing increases that began after the 1973 ban on DDT.  The historical population in North America is at estimated at 10,600 -- 12,000 breeding pairs, the majority of which (\textasciitilde75\%) bred north of $55^{\circ}$ \citep[including Greenland; ][p. 6]{cade2003return} and migrated to lower temperate and tropical latitudes. Migration counts \citep{mccarty2005using} and mid-winter counts at temperate latitudes \citep{Ydenberg2017} show strong ongoing increases of 3 -- 7 fold that began in the mid- or late 1970s; increases which have led to a current population estimate of over 60,000 falcons across North America \citep[though this is a very rough estimate][]{COSEWIC_PEFA_2017}.

This population increase includes a large increase in abundance of peregrines breeding at temperate latitudes along continental flyways, partially as a result of programs releasing captive-bred peregrines into the wild.  Releases took place throughout the continent, but the biggest programs established breeding peregrines in the Bay of Fundy, and in Delaware and Chesapeake Bays \citep[and environs; ][]{amirault20041995,Gahbauer2015,Watts2015}. Especially relevant to this paper is the introduction of peregrines into the Bay of Fundy \citep[summarized in ][]{Dekker2011}, in which 178 captive-bred birds were released 1982 -- 1993 by the Canadian Wildlife Service. The first breeding of peregrines in the region in at least a half-century (and perhaps longer; see Discussion) was recorded in 1989. Subsequently, active nest sites increased to the current level of about 35, as documented in \autoref{fig:fig-1}. Merlins (\textit{Falco columbarius}) have also become more abundant \citep{Dekker2011}, though without the aid of any reintroduction program. Watts et al. (2015) describe a very similar history in Delaware and Chesapeake Bays.  These breeding peregrines are especially significant, for they are present and actively hunting \citep{Dekker2011} throughout the sandpiper passage period, whereas migratory peregrines do not arrive until late September just as semipalmated sandpiper passage is ending \citep[see Figure 5 in ][]{lank_effects_2003}. 

As a result of these increases, southbound migration along the Atlantic coast of North America has become much more dangerous for sandpipers. We assert that the ongoing recovery of falcons and other raptors constitutes an important environmental change for many prey species that should induce strong risk effects. Demonstrated non-lethal consequences of the increased exposure to raptors include seabirds shifting to safer breeding locations in response to the recovery of bald eagles \citep[\textit{Haliaeetus leucocephalus}; ][]{MarkHipfner2012}. The increased abundance of white-tailed sea eagles (\textit{Haliaeetus alba}) in the Baltic Sea caused barnacle geese (\textit{Branta leucopsis}) to alter migration timing and to shorten the duration of parental care \citep{Jonker2010}. Falcon recovery drove Pacific dunlins (\textit{Calidris alpina pacifica}) to lose their mid-winter fat reserve, and to take up over-ocean flocking in place of roosting at high tide (\citealt{Ydenberg2010a}; see also \citealt{Dekker2011}]). Of special relevance here is the demonstration that dunlins redistributed during the non-breeding season, shifting towards greater aggregation at safer sites \citep{Ydenberg2017}. In this report we focus on semipalmated sandpipers (\textit{Calidris pusilla}) migrating southward through the Bay of Fundy. We analyze a large dataset of migratory censuses, predicting that stopover site usage has shifted toward greater use of safer sites, analogous to that found for non-breeding dunlins. As demonstrated by \citeauthor{ydenberg_western_2004} \citep[2004; ][]{pomeroy_experimental_2006}, safer sites are those at which sandpipers can feed distant from shorelines, where cover provides falcons the opportunity for stealth hunts \citep{dekker_raptor_2004}. Stealth hunts are far more successful than open hunts, which often require peregrines to undertake lengthy pursuits \citep[though see][for a contrasting prespective with multiple predators]{Cresswell2013}. 

Semipalmated sandpipers display many of the attributes expected of mortality-minimizing migrants \citep{Hope2011,Duijns2019}, and hence we expect that safety is important to their stopover decisions. Small shorebirds show a diverse range of behavioural tactics in response to predation danger \citep[e.g.,][and references therein]{lank_effects_2003,Sprague2008a,Pomeroy2008a,Fernandez2010c,beauchamp2010ethology,hilton1999choice,martins2016contrasting,ydenberg_western_2004,VandenHout2010,VandenHout2016,Cresswell2013}. Most of a migrant's time is spent at stopover sites \citep{Hedenstrom1997}, foraging to acquire the fuel necessary for extended flights \citep{Houston1998,Cimprich2005a}. From the Bay of Fundy, many semipalmated sandpipers make a long (\textasciitilde4000 km or more) trans-Atlantic flight to South America, acquiring a fuel load nearly equal to their (lean) body mass to do so. The intensive foraging required to accumulate the fuel load compromises vigilance level \citep{Beauchamp2014}, and a large fuel load makes migrants more vulnerable to predator attack. Characteristics of stopover sites, such as the amount of concealment cover available to predators, or the distance between this cover and the feeding sites used by shorebirds, make the intrinsic danger of some sites higher than others \citep{lank_ydenberg2003}. 

There are hundreds of potential stopover sites along the Atlantic coast, and in selecting stopovers shorebirds must balance the risk of predation with the benefits of good foraging conditions. In general, it appears that safety and food trade off at stopover sites so safe sites with high food availability are rare or non-existent. Previous studies have demonstrated that migrating sandpipers avoid sites that do not provide some element of food and safety \citep{Sprague2008a,Pomeroy2008a}. An increase in predator abundance is expected to shift the balances of these risks and rewards, leading to a shift away from dangerous stopover sites and towards safer ones \citep{Hope2018a}. 

Conditions other than the level of predation danger have also changed over recent decades for migrant sandpipers. These include the degradation of existing and the appearance of new habitats \citep{Iwamura2013,Studds2017,Taft2006a,Alves2012}, climate change \citep{Both2007,Gordo2007,cox2010bird,Sutherland2015,Mann2017}, and possible strong population reductions \citep{munro2017s,rosenberg2019decline}. These changing conditions could also affect stopover usage by alterations in site characteristics, energy requirements, or the degree of competition. 

We use a large survey dataset of counts of southbound semipalmated sandpipers at Bay of Fundy stopover sites to determine if semipalmated sandpipers changed site usage between 1974 and 2017. We developed an index to describe the annual distribution of birds and utilized statistical methods and simulations to diagnose whether semipalmated sandpipers adjusted stopover site selection as predator abundance increased. We predict a shift in bird usage towards safer sites as migrants increasingly had to prioritize safety. 
